\chapter*{\Large \center Abstract}

Speaker Diarization is commonly known as the task of finding out ``who spoken when?" in an audio recording. It is an important field because it is a crucial preprocessing step for many other areas in speech technology. One of the key challenges faced in this field is how to deal with domain variation. There is a lot of speaker, channel and environment variability that exists in the speech signal. Most previous diarization research has focused on specific domains and performance on diverse datasets is expected to be poor. DIHARD is a challenge that has been created by the diarization research community to boost research in the area of diverse datasets. The main aim of the project is to build a complete speaker diarization system that works within the rules of the 2019 DIHARD challenge using the Kaldi toolkit. The system is not officially submitted to the challenge because the dissertation timeline did not allow it. Several different system configurations are explored in the project. The best system involved concatenating two different speaker embeddings into a single embedding. This resulted in a diarization error rate (DER) of 24.64\% on the evaluation set, almost a 7.3\% relative improvement compared to the baseline which was at 26.58\%.

