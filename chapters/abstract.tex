\chapter*{\Large \center Abstract}

Speaker Diarization is the task of finding out ``who spoken when?" given an audio recording. It is an important field because it is a crucial preprocessing step for many areas in speech processing like speech recognition. Over recent years, the availability of fast computing power and emergence of massive amounts of multimedia data has boosted the field. The demand for various kinds of speech technology, and hence diarization, has become greater than ever. The performance of diarization systems has also improved significantly in the past decade or so thanks to the continued efforts of the research community. But even so, the absolute numbers tell a different story and it seems that there is still a long way to go. There is still a need for big improvements so that speaker diarization technology can be deployed in real world applications. This makes it an exciting research area which has a lot of potential to improve in the next few years.

