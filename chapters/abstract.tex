\chapter*{\Large \center Abstract}

Speaker Diarization is commonly known as the task of finding out ``who spoken when?" in an audio recording. It is an important field because it is a crucial preprocessing step for many other areas in speech technology. One of the key challenges faced in this field is how to deal with domain variation - the speaker, channel and environment variability that exists in the speech signal. DIHARD is a challenge that has been created by the diarization community to boost research so this problem can be solved.
The main aim of the project is to build a complete speaker diarization system in Kaldi that works within the rules of the 2019 DIHARD challenge. Several different system configurations are explored in the project. The best system involved concatenating two different speaker embeddings into a single embedding. This resulted in a DER of 24.64\%, almost 2\% less compared to the baseline at 26.58\%.

