\chapter{Appendix B: Domains and Sources}

\section{Domains}
\begin{itemize}
	\item \emph{Audiobooks}
	
	Excerpts from recordings of speakers reading aloud passages from public domain English language texts. The recordings were selected from LibriVox and each recording consists of a single, amateur reader. Care was taken to make sure that the chapters and speakers drawn from were not present in LibriSpeech, which also draws from LibriVox.
	
	\item \emph{Broadcast interview}
	
	Student-lead radio interviews conducted during the 1970s with popular figures of the era (e.g., Ann Landers, Mark Hamill, Buckminster Fuller, and Isaac Asimov). The recordings are selected from the unpublished LDC YouthPoint corpus.
	
	\item \emph{Child language}
	
	Excerpts from day long recordings of infant (6 to 18 months) speech. All audio was recorded in the home using a LENA recording device, which consists of a vest worn by the child into which a microphone has been sewn. Because of their age, the child “speech” consists of a mixture of simplistic speech consisting of short utterances (possible very disfluent), babbling, laughing, crying, and diverse uncategorizeable non-speech vocalizations. Other speakers may be present in the recording, typically one or more parents, but also siblings, friends of siblings, aunts and uncles, and adult friends of the parents. Some of the recordings have quiet backgrounds, while others have radios or televisions playing. All recordings were taken from the SEEDLingS corpus.
	
	\item \emph{Clinical}
	
	Recordings of Autism Diagnostic Observation Schedule (ADOS) interviews conducted to identify whether a child fit the clinical diagnosis for autism. ADOS is a roughly hour long semi-structured interview in which clinicians attempt to elicit language that differentiates children with Autism Spec- trum Disorder from those without (e.g., “What does being a friend mean to you?”). The children included in this collection ranged from 12-16 years in age and exhibit a range of diagnoses from autism to non-autism language disorder to ADHD to typically developing. Interviews are typically recorded for quality assurance purposes; in this case, the recording was conducted using a ceiling mounted microphone. The recordings are selected from the unpublished LDC ADOS corpus.
	
	\item \emph{Courtroom}
	
	Recordings of oral arguments from the 2001 term of the U.S. Supreme Court. The original recordings were made using individual table-mounted microphones, one for each participant, which could be switched on and off by the speakers as appropriate. The outputs of these microphones were summed and recorded on a single-channel reel-to-reel analogue tape recorder. All recordings taken from SCOTUS, an unpublished LDC corpus.
	
	\item \emph{Map task}
	
	Recordings of speakers engaged in a map task. Each map task session contains two speakers sitting opposite one another at a table. Each speaker has a map visible only to him and a designated role as either “Leader” or “Follower”. The Leader has a route marked on his map and is tasked with communicating this route to the Follower so that he may precisely reproduce it on his own map. Though each speaker was recorded on a separate channel via a close-talking microphone, these have been mixed together for the DIHARD releases. The recordings are drawn from the DCIEM Map Task Corpus (LDC96S38).
	
	\item \emph{Meeting}
	
	Recordings of meetings containing between 3 and 7 speakers. The speech in these meetings is highly in- teractive in nature consisting of large amounts of spontaneous speech containing frequent interruptions and overlapping speech. For each meeting a single, centrally located distant microphone is provided, which may exhibit excessively low gain. For the development set, these meetings are drawn from RT04, while for the evaluation set they are drawn from ROAR.
	
	\item \emph{Restaurant}
	
	Informal conversations recorded in restaurants using binaural microphones. Each session contains between 4 and 7 speakers seated at the same table at a restaurant at lunchtime and was recorded from a binaural microphone worn by a designated facilitator; the mix of the two channels recorded by this microphone are provided. This data exhibits the following properties, which are expected to make it particularly challenging for automated segmentation and recognition:
	– due to the microphone setup, the majority of the speakers are farfield
	– background speech from neighboring tables is often present, sometimes at levels close to that of
	the primary speakers in the conversation
	– background noise is abundant with clinking silverware, moving chairs/tables, and loud music all common
	– the conversations are informal and highly interactive with interruptions and frequent overlapped speech
	All data is taken from LDC’s unpublished CIR corpus.
	
	\item \emph{Sociolinguistic field recordings}
	
	Sociolinguistic interviews recorded under field conditions. Recordings consists of a single interviewer attempting to elicit vernacular speech from an informant during informal conversation. Typically, interviews were recorded in the home, though occasionally they were recorded in a public location such as a park or cafe. The development set recordings were drawn from SLX and the evaluation set from DASS.
	
	\item \emph{Sociolinguistic lab recordings}
	
	Sociolinguistic interviews recorded under quiet conditions in a controlled environment. All data is taken from the PZM microphones of LDC’s Mixer 6 collection (LDC23013S03).
	
	\item \emph{Web video}
	
	English and Mandarin amateur videos collected from online video sharing sites (e.g., YouTube and Vimeo). This domain is expected to be particularly challenging as the videos present a diverse set of topics and recording conditions; in particular, many videos contain multiple speakers talking in a noisy environment, where it can be difficult to distinguish speech from other kinds of sounds. All data is selected from LDC’s VAST collection.

\end{itemize}

\section{Sources}

\begin{itemize}
\item \emph{ADOS}

ADOS is an unpublished LDC corpus consisting of transcribed excerpts from ADOS interviews con- ducted at the Center for Autism Research (CAR) at the Children’s Hospital of Philadelphia (CHOP). All interviews were conducted at CAR by trained clinicians using ADOS module 3. The interviews were recorded using a mixture of cameras and audio recorded from a ceiling mounted microphone. Portions of these interviews determined by a clinician to be particularly diagnostic were then segmented and transcribed.
Note that in order to publish this data, it had to be de-identified by applying a low-pass filter to regions identified as containing personal identifying information (PII). Pitch information in these regions is still recoverable, but the amplitude levels have been reduced relative to the original signal. Filtering was done with a 10th order Butterworth filter with a passband of 0 to 400 Hz. To avoid abrupt transitions in the resulting waveform, the effect of the filter was gradually faded in and out at the beginning and end of the regions using a ramp of 40 ms.

\item \emph{CIR}

Conversations in Restaurants (CIR) is a collection of informal speech recorded in restaurants that LDC originally produced for the NSF Hearables Challenge, an NSF-sponsored challenge designed to promote the development of algorithms or methods that could improve hearing in a noisy setting. It consists of conversations between 3 and 6 speakers, all LDC or Penn employees, seated at the same table at a restaurant near the University of Pennsylvania campus. Recording sessions were held at lunch time using a rotating list of restaurants exhibiting diverse acoustic environments and typically lasted 60-70 minutes. All recordings were conducted using binaural microphones mounted on either side of one speaker’s head.
A limited number of regions from one recording were found to contain PII. These regions were de- identified using the same low-pass filtering approach as in ADOS

\item \emph{DASS}

The Digital Archive of Southern Speech, or DASS, is a corpus of interviews (each lasting anywhere from 3 to 13 hours) recorded during the late 60s and 70s in the Gulf Coast region of the United States. It is part of the larger Linguistic Atlas of the Gulf States (LAGS), a long-running project that attempted to preserve the speech of a region encompassing Louisiana, Alabama, Mississippi, and Florida as well as parts of Texas, Tennessee, Arkansas, and Georgia. Each interview was conducted in the field by a trained interviewer, who attempted to elicit conversation about common topics like family, the weather, household articles, agriculture, and social connections. It is distributed by LDC as LDC2012S03 and LDC2016S05.
Due to the nature of the interviews, they sometimes contain PII or sensitive materials. All such regions have been replaced by tones of matched duration. Unfortunately, this process does not appear to have been systematic, with the result that the type of tone (pure or complex), power, and frequency differs across the corpus.

\item \emph{DCIEM}

The DCIEM Map Task Corpus (LDC96S38) is a collection of recordings of two-person map tasks recorded for the DCIEM Sleep Deprivation Study. This study was conducted by the Defense and Civil Institute of Environmental Medicine (Department of National Defense, Canada) to evaluate the effect of drugs on performance degradation in sleep deprived individuals. Three drug conditions (Modafinil vs. Amphetamine vs. placebo) were crossed with three sleep conditions (18 hours vs. 48 hours vs. 58 hours awake). During each session, subjects performed a battery of neuropsychological tests (e.g., tracking tasks, time estimation tasks, attention-splitting tasks), questionnaires, and a map task. All audio was recorded via close-talking microphones under quiet conditions.

\item \emph{LibriVox}

LibriVox is a collection of public domain audiobooks read by volunteers from around the world. It consists of more than 10,000 recordings in 96 languages. Portions have previously appeared in the popular LibriSpeech corpus, though care was taken to ensure that DIHARD did not select from this subset.

\item \emph{MIXER6}

Mixer 6 (LDC2013S03) is a large-scale collection of English speech across multiple environments, modalities, degrees of formality, and channels that was conducted at LDC from 2009 through 2010. The collection consists of interviews with 594 native speakers of English spanning 1,425 sessions, each roughly 40-45 minutes in duration. Each session contained multiple components (e.g., informal conversation styled after a sociolinguistic interview or transcript reading) and was captured by a variety of microphones, including lavalier, head-mounted, podium, shotgun, PZM, and array microphones. While the corpus was released without speaker segmentation or transcripts, a portion of the corpus was subsequently transcribed at LDC. DIHARD II draws its selections from this subset.

\item \emph{ROAR}

ROAR is a collection of multiparty (3 to 6 participant) conversations recorded by LDC as part of the DARPA ROAR (Robust Omnipresent Automatic Recognition) project in Fall 2001. While portions of this collection have previously been exposed during the NIST RT evaluations, all DIHARD data comes from previously unexposed meetings. The meetings were recorded at LDC in a purpose built room using a combination of lavalier, head mounted, omnidirectional, PZM, shotgun, podium, and array microphones. For each meeting, a single centrally located distant microphone is provided.

\item \emph{RT04}

RT04 consists of meeting speech released as part of the NIST Spring 2004 Rich Transcription (RT-04S) Meeting Recognition Evaluation development and evaluation sets. This data was later re-released by LDC as LDC2007S11 and LDC2007S12. It consists of recordings of multiparty (3 to 7 participant) meetings held at multiple sites (ICSI, NIST, CMU, and LDC), each with a different microphone setup. For DIHARD, a single channel is distributed for each meeting, corresponding to the RT-04S single distant microphone (SDM) condition. Audio files have been trimmed from the original recordings to the 11 minute scoring regions specified in the RT-04S un-partitioned evaluation map (UEM) files8.

\item \emph{SCOTUS}

SCOTUS is an unpublished LDC corpus consisting of oral arguments from the 2001 term of the U.S. Supreme Court. The recordings were transcribed and manually word-aligned as part of the OYEZ project, then forced aligned and QCed at LDC.

\item \emph{SEEDLingS}

SEEDLingS is a corpus of child speech collected at the University of Rochester. Excerpts from day-long recordings conducted in the home were selected, then segmented and transcribed by LDC.

\item \emph{SLX}

SLX (LDC2003T15) is a corpus of sociolinguistic interviews conducted in the 1960s and 1970s by Bill Labov and his students. The interview subjects range in age from 15 to 81 and represent a diverse sampling of ethnicities, backgrounds, and dialects (e.g., southern Amercian English, African American English, northern England, and Scotland). While the recordings have good sound quality for field recordings (especially from that era), they were collected in a range of environments ranging from noisy homes (e.g., small children running around in the background) to public parks to gas stations.

\item \emph{VAST}

The Video Annotation for Speech Technologies (VAST) corpus is a (mostly) unexposed collection of approximately 2,900 hours of web videos (e.g., YouTube and Vimeo) intended for development and evaluation of speech technologies; in particular, speech activity detection (SAD), diarization, language identification (LID), speaker identification (SID), and speech recognition (STT). Collection emphasized videos where people are talking with a particular emphasis on videos where the speakers spoke primarily English, Mandarin, and Arabic, which comprise the bulk of the corpus9. Portions of this corpus have been exposed previously as part of the NIST 2017 Speech Analytic Technologies Evaluation, the NIST 2017 Language Recognition Evaluation, NIST 2018 Speaker Recognition Evaluation, and DIHARD I.

\item \emph{YouthPoint}

YouthPoint is an unpublished LDC corpus consisting of episodes of YouthPoint, a late 1970s radio program run by students at the University of Pennsylvania. The show had an interview format similar to shows such as NPR’s Fresh Air and consisted of interviews between University of Pennsylvania students and various popular figures. The recordings were conducted in a studio on open reel tapes and later digitized and transcribed at LDC.

\end{itemize}