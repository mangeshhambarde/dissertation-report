\chapter{Introduction}

Speaker diarization is the task of finding out ``who spoke when?" in an audio recording with an unknown amount of speakers. It aims to find all segments of speech within the recording, possibly overlapping, along with their intra-recording speaker identities. It acts as an important upstream preprocessing step for most tasks in speech processing, like speech recognition, speech enhancement, speech coding etc.

With increase in computing power, speech processing technologies have achieved incredible advances in the past decade that were not possible earlier. This has increased interest in Rich Transcription (RT) technologies that can be used to automatically index the enormous amount of audio and video information that is generated in the modern world. Since speaker diarization is an important part in any RT system, there is a great deal of research interest in the area.

Diarization is not an easy problem since the output is affected by several factors like the application domain (broadcast news, meetings, telephone audio, internet audio, restaurant speech, clinical recordings etc), types and quality of microphones used (boom, lapel, far-field), inter-channel synchronization problems, overlapping speech, etc. These days, most of the research focuses on the meeting speech domain, since most problems that exist in speech recognition are encountered in this domain. The meeting scenario is thus often termed as ``speech recognition complete".

The DIHARD challenge was created to establish standard datasets for diarization and create performance baselines for comparison, thus encouraging further research. The challenge focuses on ``hard" diarization, combining several domains of speech like broadcast speech, meeting speech, telephone speech, and many more. Creating a system for the challenge can be a rewarding experience since it gives a chance to learn about state-of-the-art speaker diarization techniques.

\section{Aims and Objectives}

The aim of this project is to build a competent system for the DIHARD II challenge. It is expected that this exercise will greatly help in acquiring a good understanding of the field of speaker diarization and thus be extremely rewarding.

This project is concerned with the second iteration of the challenge, DIHARD II. There already exist published results for the DIHARD I systems which can be used as a reference. A few of the top performing systems from DIHARD I are chosen as inspiration.

\section{Overview of the Report}

Chapter 2 gives a brief overview of the speaker diarization field by going through the modules of a typical diarization system one-by-one. Chapter 3 describes the tasks the project is divided in, how the project will be evaluated and the conditions of evaluation imposed by DIHARD. In the end it also discusses ethical, professional and legal issues associated with the project. Chapter 4 analyses the risks associated with the project, quantifies them, and presents actions that can be taken to mitigate those risks. It also give a project plan and a Gantt Chart showing the timeline of the tasks of the project. Finally Chapter 5 presents a summary of conclusions.